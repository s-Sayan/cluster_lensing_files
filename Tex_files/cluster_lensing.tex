
\documentclass[12pt,letterpaper]{article}	
\UseRawInputEncoding
% PERSONAL PACKAGES [add below] 
\usepackage{expex,marvosym,tabularray,xcolor}
%\usepackage{jcappub}
\usepackage{times} %obsolete, but works for the font and style
\usepackage{tipa}
%\usepackage{natbib}
% 	\setcitestyle{semicolon,aysep={},yysep={,},notesep={:}}
 	%see below for instructions on natbib bibliography
%\usepackage[square,sort,comma,numbers]{natbib}
\usepackage{lipsum} % this and the following package and the settings beneath them are for maintaining indentation and still having ragged-right alignmnent
\usepackage{ragged2e}
\usepackage{amsmath}
\numberwithin{equation}{section}
\setlength\RaggedRightParindent{0.3in}
\RaggedRight

%\usepackage[colorlinks=true,allcolors=blue]{hyperref}%

\usepackage{scrextend} % this package and the following settings are for the footnote formatting
\deffootnote[.5em]{0em}{1em}{\textsuperscript{\thefootnotemark}\,}

\newcounter{savefootnote}   % these settings allow the use of an asterisk as a footnote label (for the author line below the title.
\newcounter{symfootnote}
\newcommand{\symfootnote}[1]{%
   \setcounter{savefootnote}{\value{footnote}}%
   \setcounter{footnote}{\value{symfootnote}}%
   \ifnum\value{footnote}>8\setcounter{footnote}{0}\fi%
   \let\oldthefootnote=\thefootnote%
   \renewcommand{\thefootnote}{\fnsymbol{footnote}}%
   \footnote{#1}%
   \let\thefootnote=\oldthefootnote%
   \setcounter{symfootnote}{\value{footnote}}%
   \setcounter{footnote}{\value{savefootnote}}%
}

\def\Eq#1{eq.~\eqref{#1}}

\usepackage[labelsep=period]{caption}

\usepackage[margin=1.0in]{geometry}
\usepackage[compact]{titlesec}
	\titleformat{\section}[runin]{\normalfont\bfseries}{\thesection.}{.5em}{}[.]
	\titleformat{\subsection}[runin]{\normalfont\scshape}{\thesubsection.}{.5em}{}[.]
	\titleformat{\subsubsection}[runin]{\normalfont\scshape}{\thesubsubsection.}{.5em}{}[.]
%\usepackage[usenames,dvipsnames]{color}	
\usepackage[colorlinks,allcolors={blue},urlcolor={blue}]{hyperref} 

%%% PERSONAL DEFINITIONS (if needed)

%%% LSA DEFINITIONS

% for the abstract
% Abstracts have to be 12pt, indented 1.4 inches on each side, and inline with the label
\renewenvironment{abstract}{%
\noindent\begin{minipage}{1\textwidth}
\setlength{\leftskip}{0.4in}
\setlength{\rightskip}{0.4in}
\textbf{Abstract.}}
{\end{minipage}}

% keywords environment
\newenvironment{keywords}{%
\vspace{.5em}
\noindent\begin{minipage}{1\textwidth}
\setlength{\leftskip}{0.4in}
\setlength{\rightskip}{0.4in}
\textbf{Keywords.}}
{\end{minipage}}

%%% MAIN DOCUMENT
\begin{document} 

%%If using linguex, need the following commands to get correct LSA style spacing
%% these have to be after  \begin{document}
%\setlength{\Extopsep}{6pt}
%\setlength{\Exlabelsep}{9pt} %effect of 0.4in indent from left text edge

%title and author lines
\begin{center}
\normalfont\large\bfseries
Project Report: CMB Weak Lensing by Galaxy-clusters
\vskip .5em
\normalfont
%{\symfootnote{ }}
\vskip .5em
\end{center}

%\begin{abstract}

%\end{abstract}

%\begin{keywords} %Separated by semicolons

%\end{keywords}

\section{Introduction}

The weak gravitational lensing of the Cosmic Microwave Background (CMB) anisotropies is now a well-known phenomenon that has been observed in recent high-resolution ground-based telescopes like Atacama Cosmology Telescope (ACT), the South Pole Telescope (SPT), or the satellite Planck. The lensing potential profile reconstructed from the lensed CMB contains a wealth of information about the late time universe. In this project, we plan to develop an improved method to capture the lensing information of the CMB on small, arcminute scales. The algorithm is based on a maximum likelihood or a posteriori reconstruction of the lensing signal, which is known to work well on larger scales. An important application of our method is the measurement of the mass profile of galaxy clusters. Galaxy clusters are one of the most interesting probes of dark energy and accurate measurements of their masses are of great interest to test cosmological models. With analytical and numerical tools we will gain a detailed understanding of how much better our methods can perform applied to present and future data.
\section{Theory}
\subsection{\textbf{Ingredients of the cluster mass-profile}}
The Galaxy clusters has been observed with their SZ signal in few arcmin scales. For the mass profile of the cluster, there are several models in the literature. But the most conventional one is the Navarro-Frenk-White (NFW) profile \cite{Navarro:1995iw} for dark matter halos 
\begin{align}
    \rho(r) = \begin{cases}
               \frac{\rho_0}{(1+\frac{r}{r_s})^2} &\quad \text{if} \ r < R_{\text{trunc}} \\
               0  &\quad \text{if} \ r > R_{\text{trunc}}
               \end{cases},
\end{align}
where $\rho_0$ is called the characteristics cluster density, $r_s$ is called the scale radius and $R_{\text{trunc}}$ is called the truncation radius, beyond which the density profile is taken to be as zero as a more realistic model. The mass of the cluster is characterized by  $M_{200}$, i.e. the mass enclosed in a sphere of radius $R_{200}$. $R_{200}$ is a radius of a hypothetical sphere within which the average density of the cluster is 200 times the critical density of the universe $\rho_{\text{crit}}$ at the cluster redshift $z$. Using this condition, one can write the following relation between $\rho_0$ and $\rho_{\text{crit}}$,
\begin{align}
    &M_{200} = \int_0^{R_{200}}\rho(r) 4\pi r^2 dr \\
    &\frac{200}{3}\rho_{\text{crit}}(z)R_{200}^3 = r_s^3 \rho_0 \left[ \ln{\left(\frac{R_{200}+r_s}{r_s}\right)} - \left(\frac{R_{200}}{R_{200} + r_s}\right) \right] \\
    &\rho_0 = \rho_{\text{crit}}(z) \frac{200}{3}\frac{c_{200}}{\ln{(1+c_{200})}-\left(\frac{c_{200}}{1+c_{200}}\right)}\\
    &\rho_0 = \frac{M_{200}}{4\pi r_s^3 \left[\ln{(1+c_{200})}-\left(\frac{c_{200}}{1+c_{200}}\right)\right]}
\end{align}

Any of these equation can be useful for the relation with $\rho_0$. Here, the $c_{200}$ is called the concentration parameter, $c_{200} = R_{200}/r_s$. $c_{200}$ has following dependence on mass ($M_{200}$) and redshift ($z$) \cite{Geach:2017crt, Duffy:2008pz}
\begin{align}
    c_{200}(M_{200}, z) = 5.71(1+z)^{-0.47}\left(\frac{M_{200}}{2\times10^{12}h^{-1}M_{\odot}}\right)^{-0.084}
\end{align}\label{c200}
According to the definition of $M_{200}$ and $R_{200}$,
\begin{align}
    M_{200} = \frac{4}{3}\pi R_{200}^3 \times 200\rho_{crit}(z)
\end{align}\label{R_200}
Given the value of $M_{200}$ and $z$, we can always calculate the $R_{200}$ from eq. (2.7) and $c_{200}$ from eq. (2.6) and finally $r_{s}$ from the definition of $c_{200}$.
\subsection{\textbf{Lensing by the Cluster}}
The most relevant quantity that we shall work on is the convergence $\kappa (\hat{n})$, which is related to the lensing potential as
\begin{align}
    \kappa(\hat{n}) = -\frac{1}{2}\nabla^2_{\hat{n}}\phi(\hat{n})
\end{align}

The cluster convergence is given by
\begin{align}
    \kappa_{cl}(r) = \frac{\Sigma_{cl}(r)}{\Sigma_{crit}(z)},
\end{align}
where $\Sigma_{cl}(r)$ is the projected surface density of the cluster:
\begin{align}
    \Sigma_{cl}(R) = 2 \int_{R}^{R_{\text{trunc}}}\frac{r\rho(r)}{\sqrt{r^2-R^2}}dr
\end{align},
and $\Sigma_{\text{crit}}(z)$ is the critival surface density of the universe at the cluster redshift:
\begin{align}
    \Sigma_{\text{crit}}(z) = \frac{c^2}{4\pi G}\frac{d_{A,CMB}}{d_{A,cl}d_{A,cl-CMB}}
\end{align}
In literature, there are mainly two choice of the profile regarding truncation. One 

\label{Bibliography}

%\lhead{\emph{Bibliography}} % Change the page header to say "Bibliography"
%\bibliographystyle{unsrt} % Use the "unsrtnat" BibTeX style for formatting the Bibliography
\bibliographystyle{JHEP}
\bibliography{references} % your bib file
\end{document}